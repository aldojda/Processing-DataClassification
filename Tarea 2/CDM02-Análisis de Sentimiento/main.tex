\documentclass[a4paper,12pt]{article}

\usepackage[utf8]{inputenc} % Para poder escribir caracteres especiales
\usepackage[spanish]{babel} % Para configurar el idioma en español
\usepackage{graphicx} % Para incluir gráficos
\usepackage{amsmath} % Para ecuaciones matemáticas
\usepackage{hyperref} % Para enlaces y referencias
\usepackage{geometry} % Para definir márgenes
\geometry{margin=1in} % Márgenes de 1 pulgada
\usepackage{subcaption} % Para las subfiguras

\title{Tarea 2:  Análisis sentimiento a comentarios de Misprofesores.com}
\author{Aldo Daniel Ojeda Rodriguez}
%\date{\today}

\begin{document}

\maketitle

\begin{abstract}
%--------------------
    % Resumen de la tarea. Aquí describes brevemente el propósito y los principales hallazgos de tu trabajo.
 %--------------------
\end{abstract}

\section{Introducción}

Para el presente proyecto, se desarrolló un código que aplica técnicas de web scraping a datos obtenidos del portal \textbf{MisProfesores} el cual es una plataforma ampliamente utilizada por estudiantes de diversas instituciones mexicanas, tanto a nivel bachillerato como universitario, para consultar y evaluar la calidad de los profesores antes de decidir qué materias seleccionar.

El objetivo principal de este proyecto es extraer y analizar las opiniones y calificaciones de los profesores disponibles en \textbf{MisProfesores}, con un enfoque especial en el análisis de sentimiento. Esto permitirá identificar patrones en las percepciones de los estudiantes y poder generar un ranking de aquellas materias que contienen comentarios positivos y negativos.

\section{Metodología}

La metodología empleada para este proyecto incluye la utilización de técnicas avanzadas de web scraping para recopilar datos de manera automatizada de donde se seleccionaron 5 materias impartidas en la Facultad de Ciencias Físico Matemáticas: 
\begin{center}
 \begin{tabular}{lr}
geometria analitica & 73 \\
calculo integral & 72 \\
algebra & 63 \\
calculo diferencial & 54 \\
fisica basica & 46 \\
calculo de varias variables & 41 \\
\end{tabular}
\end{center}

enseguida se aplicó un modelo tipo transformer del tipo \textbf{bert-base-multilingual-uncased-sentiment} entrenado para clasificar los comentarios de los estudiantes  en una escala del 1 al 5, donde 1 indica una opinión negativa y 5 representa una opinión positiva.

\section{Resultados y Discusión de Resultados}
\begin{center}
\begin{tabular}{lrrrrr}
    Material & 1 & 2 & 3 & 4 & 5 \\
    algebra & 3.17\% & 9.52\% & 12.70\% & 44.44\% & 30.16\% \\
    calculo de varias variables & 12.20\% & 17.07\% & 14.63\% & 34.15\% & 21.95\% \\
    calculo diferencial & 14.81\% & 16.67\% & 24.07\% & 29.63\% & 14.81\% \\
    calculo integral & 13.89\% & 16.67\% & 15.28\% & 22.22\% & 31.94\% \\
    fisica basica & 17.39\% & 15.22\% & 10.87\% & 21.74\% & 34.78\% \\
    geometria analitica & 21.92\% & 20.55\% & 13.70\% & 21.92\% & 21.92\% \\
\end{tabular}
\end{center}

La tabla muestra la distribución porcentual de opiniones sobre materias, en una escala del 1 al 5 (1 negativo, 5 positivo). En Álgebra predominan opiniones positivas (44.44\% en nivel 4, 30.16\% en nivel 5), con pocas negativas. Cálculo de Varias Variables tiene una distribución más equilibrada, con opiniones positivas y negativas notables. Cálculo Diferencial y Geometría Analítica muestran una mayor dispersión de opiniones por otro lado Cálculo Integral destaca con un 31.94\% de opiniones muy positivas, y Física Básica tiene un alto porcentaje de opiniones positivas (34.78\% en nivel 5) pero también significativas opiniones negativas.

En resumen, Álgebra y Física Básica tienen opiniones predominantemente positivas, mientras que Cálculo de Varias Variables y Geometría Analítica muestran una mayor diversidad en las calificaciones y por otro lado la materia peor calificada seria calculo diferencial.

\subsection{Conclusión}

Un clasificador de sentimiento nos permite poder interpretar los sentimientos de los usuarios y darles una referencia para comparar o describir ciertos eventos o muestras de datos, en este caso el análisis de sentimiento permitió seleccionar las materias con un sentimiento generalizado positivo y distinguirla de aquellas materias que presentan sentimientos generalizados negativos.



%\section{Referencias}
 %   Lista de las referencias bibliográficas utilizadas en tu trabajo. Puedes utilizar el siguiente formato para las referencias en formato BibTeX.

%\begin{thebibliography}{9}
 %   \bibitem{Referencia1}
  %  Autor1, Autor2,
  %  \textit{Título del artículo o libro},
  %  Revista o Editorial,
   % Año.

    %\bibitem{Referencia2}
    %Autor1, Autor2,
    %\textit{Título del artículo o libro},
    %Revista o Editorial,
    %Año.
%\end{thebibliography}

\end{document}
